\documentclass{article}
\usepackage[UTF8]{ctex}
\usepackage{geometry}
\usepackage{multirow}
\usepackage{natbib}
\geometry{left=3.18cm,right=3.18cm,top=2.54cm,bottom=2.54cm}
\usepackage{graphicx}
\pagestyle{plain}	
\usepackage{setspace}
\usepackage{enumerate}
\usepackage{caption2}
\usepackage{datetime} %日期
\renewcommand{\today}{\number\year 年 \number\month 月 \number\day 日}
\renewcommand{\captionlabelfont}{\small}
\renewcommand{\captionfont}{\small}
\begin{document}

\begin{figure}
    \centering
    \includegraphics[width=8cm]{upc.png}

    \label{figupc}
\end{figure}

	\begin{center}
		\quad \\
		\quad \\
		\heiti \fontsize{45}{17} \quad \quad \quad 
		\vskip 1.5cm
		\heiti \zihao{2} 《计算科学导论》个人职业规划
	\end{center}
	\vskip 2.0cm
		
	\begin{quotation}
% 	\begin{center}
		\doublespacing
		
        \zihao{4}\par\setlength\parindent{7em}
		\quad 

		学生姓名:\underline{\qquad  万舒成 \qquad \qquad}

		学\hspace{0.61cm} 号:\underline{\qquad 2007010221\qquad}
		
		专业班级:\underline{\qquad 本研人工智能2001 \qquad  }
		
        学\hspace{0.61cm} 院:\underline{计算机科学与技术学院}
% 	\end{center}
		\vskip 1.5cm
		\centering
		\begin{table}[h]
            \centering 
            \zihao{4}
            \begin{tabular}{|c|c|c|c|c|c|c|c|c|}
            % 这里的rl 与表格对应可以看到,姓名是r,右对齐的;学号是l,左对齐的;若想居中,使用c关键字。
                \hline
                \multicolumn{5}{|c|}{分项评价} &\multicolumn{2}{c|}{整体评价}  & 总    分 & 评 阅 教 师\\
                \hline
                自我 & 环境 & 职业 & 实施 & 评估与 & 完整性 & 可行性 &\multirow{2}*{} &\multirow{2}*{}\\
                分析& 分析& 定位 & 方案 & 调整 & 20\% & 20\% & ~&~ \\\            
                10\% & 10\% & 15\% & 15\% & 10\% & &  &~ &~\\
                \cline{1-7} 
                & & & & & & & ~&~ \\
                & & & & & & & ~&~ \\
                \hline      
            \end{tabular}
        \end{table}
		\vskip 2cm
		\today
	\end{quotation}

\thispagestyle{empty}
\newpage
\setcounter{page}{1}
% 在这之前是封面,在这之后是正文
\section{自我分析}
\subsection{自然条件}
本人为男性,现在18岁,刚刚成年,正是年轻力壮的时候,身体健康,没有太大的毛病,也没有重大疾病史,居住在江西省抚州市黎川县,是一个经济不太发达的县,与外界的联系只有国道和高速公路,没有铁路。户口为农村户口。
\subsection{性格分析}
小时候受家里的影响,脾气比较暴躁,遇到解决不了的事情心情就会不好,脾气会很大。但是与他人相处时比较和善,不会轻易发脾气,和他人交谈时也总是满脸笑容。工作任务性强,喜欢完成下发的任务,而不喜欢去主动学习其他的知识去拓宽自己的知识领域。做事有一定的判断能力,能依靠自己解决问题。性格介于内向与外向之间,父母和姐姐都是很外向的人,但我小时候宅在家里之和父母和姐姐之间沟通,也不出去结识结识其他人,导致性格比较内向,到了高中的时候就开始逐渐外向,与同学老师交往,到了大学也感觉比较活跃,与其他人相处得也挺好的,积极参与活动,在一些重要的发言场合时会比较紧张,但是不是紧张得说不出话来的那种,和父母去探望亲戚时,也总是因为叫不出称呼而总是跟在父母身边不太说话,只能一味的笑和说你好。
\subsection{教育与学习经历}
幼儿园,小学,初中都是在乡下上的学,高中时在县里第一的中学上学,并且进入实验班学习,大学时进入中国石油大学(华东)。从小时候开始到现在没有上过补习班或者衔接班之类的,也没有上过那种培训技能的课堂,比如练字,乐器之类的,仅仅学习课本知识,知识领域比较局限。
\subsection{工作与社会阅历}
没有去打过暑假或者寒假工,但是到了10到3月份的时候会帮助父母搬运和组装家具,村中有什么红白事时也会去帮帮忙。到了大学时,进入餐厅当志愿者,处理一些餐厅事务,与同学们交流,实时了解同学对餐厅的意见并且反馈给经理;担任生活委员,处理同学们生活方面的问题,两周一次查寝规范同学们的宿舍内务整理准则,整理同学生日并以接龙形式传下去并提前再次提醒,管理班费;担任美食社实践部成员,在社团举办线下活动时维护现场秩序,经常性地写美食短文,向同学们推荐石大美食,在线上活动时审核同学们的打卡记录。
\subsection{知识、技能与经验}
当初选择计算机最大的原因就是因为小时候用电脑打游戏,对电脑使用得比较多,有一些小问题也能够解决,但是对于计算机方面的知识了解甚少,到了现在也就只会对word、excel等常用办公软件和一些编译器的使用,很多东西都还没有涉及,经验方面也就只有用电脑帮母亲找一些音频资源,对家里的电脑进行系统重装。
\subsection{兴趣爱好与特长}
在兴趣爱好方面,上大学以前喜欢打游戏,但是上大学以后慢慢地就不打游戏了,相对应的更喜欢听着歌发呆,偶尔嘴中再哼两句,因为小时候也就天天打游戏也没有去培养什么兴趣爱好,导致现在也没有什么东西是特别喜欢的。在特长方面,并没有什么特别厉害的地方,学习不太强,编程能力也不够,运动方面也就跑跑步,但也不如别人,就连游戏也玩得越来越菜了,多方面发展,但没有哪方面特别突出。
\section{环境分析}
\subsection{社会环境分析}
当前中美贸易之间的关系还是比较紧张,经过这次疫情的变化后,也出现了很多抵制中国的国家,国外就业还是比较麻烦。同时这次疫情也让很多公司的经济衰退,各大公司裁员,大量公司倒闭,虽然现在的经济仍然是正增长,但是在生活中还是感觉到经济状态还没有恢复得特别好。大学毕业生的数量越来越多,国家虽然增加了研究生扩招规模,但是作用还是不大,就业压力仍然很大,加上疫情期间很多公司倒闭,就业岗位大量减少,对大学生的要求也越来越高,大学生就业压力越来越大,就业形势不太明朗。
\subsection{家庭环境分析}
没有女朋友,家中经济状况一般,家人就希望能够找到一份高薪的工作,没有家族传统说必须子承父业。
\subsection{职业环境分析}
从1984年的“计算机的普及要从娃娃抓起”这句话开始,到1990年改革开放蓬勃发展,计算机成为当时最“高大上”的专业,计算机学习热潮一度火爆,从以前的人才缺口大到现在,计算机人才可以说已经开始饱和甚至溢出了,很多人都是无脑选择计算机,转行计算机的,计算机行业的竞争压力越来越大。计算机发展前景是越来越好的,但是学习计算机的人数只增不减,其他与计算机相关不大的专业也要求学习计算机。个人比较喜欢安逸一点的生活,所以毕业后想去大厂上班,暂时没有创业的打算。进入大厂后想成为一名软件工程师,打造几款软件或者对软件进行维护。近些年来,软件的范围越来越大,很多硬件都开始被软件所代替,而且很多现实生活中的功能也在网络上面有体现,比如外卖软件,打车软件等等,软件的用途和范围越来越广,发展前景也是非常广阔的。


\subsection{地域与人际环境分析}
以后想工作的城市主要在长江三角洲地区,优先考虑上海和杭州。这两个地方都属于南方地区,作为一个南方人更适应这样的环境。计算机发展时间比较长,整个城市都有很好的计算机文化氛围,城市的经济发展也比较好,计算机方面的公司多,就业岗位也多,机会大,仍然具有很好的发展前景。离家也比较近,很多亲戚都在长江三角洲地区,在这个地方工作会受到比较好的照顾,工作起来比较轻松,而且如果以后工作地点离亲戚家比较近的话,还可以住在亲戚家里,也可以节省很大一笔开支,为后期职业发展做铺垫。

\section{职业定位}
\subsection{行业领域定位与理由}
\begin{itemize}
	\item 软件工程师:软件行业发展前景好,工作岗位比较多,难度也不是很大,工作也比较稳定。更细一步理由参考上方职业环境分析中的内容。
	\item 人工智能:如果一直留在本研班且按照培养方案的进行,以后很大几率就会与人工智能打交道,从事人工智能也就顺理成章,毕竟大学学的专业就是人工智能,学的知识也是与之相关的,毕业后也有很多的理论基础去支撑着。不过我不太喜欢搞科研,只想做一个“打工人”,不过也能作为一种参考。
	\item 大数据分析:我们现在处于一个大数据的时代,很多计算机领域的内容都需要大数据作为前提,比如数据库等内容就需要大数据。大数据的发展前景也是非常好的,起步也比较晚,就业机会也比较多,有新的创新成果也相对来说更容易一点。
	\item 公务员:这么多年来,公务员都是很多大学生毕业后想考取的职业,但是这种国考真的是太火爆了,很多人,无论是985、211还是普通学校的都去报考公务员,虽然说工作稳定、轻松,但是薪水并不高,如果毕业后实在找不到工作,可以作为最后一手。
\end{itemize}
\subsection{职业岗位起点定位与理由}
	无论以后从事的是哪个职业,刚进公司肯定只能当一个很普通的程序员。没有亲戚,也没有同学在大厂里面当大官的,没有人引荐只能从基层干起,然后一步一步地向上爬, 最后可能当个项目经理之类的,管理一整个项目的研制。至于公务员也是如此,一进去可能就当个镇或者乡的公务员,之后向上爬,提高层级。

\subsection{职业目标与可行性分析}
\begin{itemize}
	\item 软件工程师:目标就是当一个软件的项目经理,负责一整个软件的研发工作,设计一款自己的软件。在大学期间,承担起啦啦操队内成员练习动作和变换队形的管理工作,指导他们完成啦啦操,一个学期下来了,搞得还算可以,发现自己也有一点管理能力, 能够承担起一个简单项目。
	\item 人工智能:目标就是当一个普通的程序员,或者和软件工程师的目标一样。
	\item 大数据分析:这个就比较常规了,当个大数据分析师,针对那些产生大量数据的行业中的公司,特别是医疗保健,制药,金融服务,物流,物联网,制造业等,工作不会太累。要求就是会数据分析和大数据编程,这些在后面都会学到。
	\item 公务员:如果以后成为公务员,就希望担任一个监察方面的官职,并且官越大越好。从小到大看了很多腐败的现象,只有自己成为监察官并且官够大,才能够在实际中遏止腐败现象,不至于束手无策,为百姓除去那些腐败的官员,给民众一个和谐的社会。
\end{itemize}
\begin{enumerate}[(1)]
	\item 短期目标(大学4年)
	 大学阶段,第一要提升的就是自己的编程能力,熟悉各种编译软件和编译语言的使用,尤其在JAVA语言方面多下功夫,熟悉运用JAVA编译程序。强化自己的社交能力,使自己完全摆脱内向的性格,练好普通话,矫正口音问题。提升自己临场能力,演讲、表演时不紧张。
	\item 中长期目标(5-10年)
	在研究生阶段和毕业的这几年时间内,主要的目标就是针对一项技术进行突破,对某一方面进行创新,有自己的闪光点。毕业后根据自己擅长的方面寻找工作。如果毕业时的就业压力很大的话可以选择先就业再择业,先在一个工作岗位上面锻炼自己,积累积累工作经验,使自我价值得到较大的提升,为以后找到理想的工作奠定基础。
\end{enumerate}

\section{实施方案}
\begin{enumerate}[1、]
	\item 珍惜石大和本研班的资源,利用这些资源去全方面提升自己的能力。我属于后劲大的人,在大学学习前期稳一点,在后期充分发挥优势突击。
	\item 学习主动性不强,可以找一个相似的人互相监督,设计打卡内容,每周打一次卡,强迫自己去完成一些内容。多多去学习课外知识,让自己忙起来,减少发呆浪费的时间。在闲暇时间可以多了解了解国家各种政策和形式,多多了解这个社会。
	\item 积极与同学和学长交流,与他们交朋友,拓展自己的人脉;借助各种渠道,把握就业机会,利用人才交流会,网络资源等途径寻找合适就业岗位,主动大胆把自己真实才干推销出去,珍惜和抓住来之不易的就业机会。
	\item 在安顿好家中的情况后再去安心工作,学习,生活方面就日常保持整洁就行。
	\item 在学习和工作期间如果感觉到无聊或者缺乏灵感动力时,可以选择看一部轻松欢乐的电影,放松放松心情,或者与自己关系好的人一起出去散个步,旅个游。
\end{enumerate}
\section{评估与调整}
\subsection{评估时间}
可以一个月评估一次,思考自己这个月来的行动是否符合自己的职业规划的要求,以便可以及时做出调整,如果在实践中发现规划有不足的地方,也可以稍作修改。
\subsection{评估内容}
可以从成果目标、经济目标、能力目标、职务目标等方面总结,确定哪些目标已按预期实现,哪些目标尚未达到,对已实现的成果总结经验,对未完成的目标分析原因。
\subsection{调整原则}
根据自己的努力状况,评估自己是否足够努力,及时调整自己状态;如果在实施过程中发现自己的想法有偏离职业规划,可以权衡一下哪个更有价值,然后修改哪个;如果社会大环境变了,可以适当地修改职业规划。




\end{document}
